\documentclass[11pt,a4paper]{article}
\usepackage[utf8]{inputenc}
\usepackage[T1]{fontenc}
\usepackage[french]{babel}
\usepackage{amsmath,amssymb}
\usepackage{booktabs}
\usepackage{siunitx}
\usepackage{tikz}
\usepackage{xcolor}
\usepackage{geometry}
\usepackage{caption}
\usepackage{fancybox}

\usetikzlibrary{patterns,arrows.meta,decorations.pathreplacing,calc,positioning,shapes.geometric}

\geometry{margin=2.5cm}

\definecolor{wpetg}{RGB}{34,139,34}      % Vert forêt pour W/PETG
\definecolor{inox}{RGB}{192,192,192}     % Gris argent pour Inox
\definecolor{water}{RGB}{100,149,237}    % Bleu pour eau
\definecolor{source}{RGB}{255,215,0}     % Or pour source
\definecolor{gamma}{RGB}{255,69,0}       % Orange-rouge pour gammas

\title{\textbf{Écran de Protection Sandwich}\\[0.5cm]
\Large W/PETG + Inox + W/PETG\\[0.3cm]
\normalsize Configuration et Résultats de Simulation Geant4}

\author{Simulation Monte-Carlo - Eu-152}
\date{\today}

\begin{document}

\maketitle

%==============================================================================
\section{Configuration Géométrique}
%==============================================================================

\begin{figure}[h!]
\centering
\begin{tikzpicture}[scale=1.0, >=Stealth]

% ============================================================================
% TITRE
% ============================================================================
\node[font=\large\bfseries] at (6,8.5) {Coupe Longitudinale du Système de Blindage};

% ============================================================================
% AXE Z (horizontal)
% ============================================================================
\draw[->,thick] (-1,0) -- (14,0) node[right] {$z$ (cm)};
\foreach \x in {0,2,4,6,8,10,12} {
    \draw (\x,0.1) -- (\x,-0.1) node[below] {\x};
}

% ============================================================================
% SOURCE Eu-152 (à z = 2 cm)
% ============================================================================
\filldraw[fill=source, draw=black, thick] (2,3.5) circle (0.3);
\node[above=0.4cm, font=\bfseries] at (2,3.5) {Source Eu-152};
\node[below=0.1cm, font=\small] at (2,3.2) {$z = 2$ cm};
\node[below=0.5cm, font=\small] at (2,3.2) {$A = 44$ kBq};

% Émission conique (60°)
\draw[gamma, thick, dashed] (2,3.5) -- (3.75,5.5);
\draw[gamma, thick, dashed] (2,3.5) -- (3.75,1.5);
\draw[gamma, thick, ->] (2,3.5) -- (3.5,3.5);
\draw[gamma, thick, ->] (2,3.5) -- (3.4,4.2);
\draw[gamma, thick, ->] (2,3.5) -- (3.4,2.8);

% Arc pour montrer l'angle
\draw[thick] (2.8,3.5) arc (0:30:0.8) node[midway, right, font=\small] {60°};
\draw[thick] (2.8,3.5) arc (0:-30:0.8);

% ============================================================================
% SANDWICH : W/PETG (7mm) + INOX (4mm) + W/PETG (7mm)
% Position : face avant z = 3.75 cm, face arrière z = 5.55 cm
% ============================================================================

% Dimensions verticales du sandwich
\def\sandwichbottom{1.0}
\def\sandwichtop{6.0}
\def\sandwichheight{5.0}

% Positions en z (échelle : 1 cm réel = 1 unité TikZ)
\def\frontface{3.75}
\def\wpetgone{4.45}      % fin W/PETG avant
\def\inoxend{4.85}       % fin Inox
\def\backface{5.55}      % fin W/PETG arrière

% --- Couche 1 : W/PETG avant (7 mm) ---
\fill[wpetg!70, draw=black, thick] 
    (\frontface,\sandwichbottom) rectangle (\wpetgone,\sandwichtop);
\node[white, font=\bfseries, rotate=90] at ({(\frontface+\wpetgone)/2},3.5) {W/PETG};

% --- Couche 2 : INOX (4 mm) ---
\fill[inox, draw=black, thick] 
    (\wpetgone,\sandwichbottom) rectangle (\inoxend,\sandwichtop);
\node[black, font=\bfseries, rotate=90] at ({(\wpetgone+\inoxend)/2},3.5) {INOX};

% --- Couche 3 : W/PETG arrière (7 mm) ---
\fill[wpetg!70, draw=black, thick] 
    (\inoxend,\sandwichbottom) rectangle (\backface,\sandwichtop);
\node[white, font=\bfseries, rotate=90] at ({(\inoxend+\backface)/2},3.5) {W/PETG};

% ============================================================================
% COTES ET DIMENSIONS
% ============================================================================

% Accolade supérieure pour épaisseur totale
\draw[decorate, decoration={brace, amplitude=8pt, raise=3pt}, thick]
    (\frontface,\sandwichtop) -- (\backface,\sandwichtop)
    node[midway, above=12pt, font=\bfseries] {18 mm};

% Cotes individuelles (en bas)
\draw[decorate, decoration={brace, amplitude=5pt, mirror, raise=3pt}, thick]
    (\frontface,\sandwichbottom) -- (\wpetgone,\sandwichbottom)
    node[midway, below=10pt, font=\small] {7 mm};

\draw[decorate, decoration={brace, amplitude=5pt, mirror, raise=3pt}, thick]
    (\wpetgone,\sandwichbottom) -- (\inoxend,\sandwichbottom)
    node[midway, below=10pt, font=\small] {4 mm};

\draw[decorate, decoration={brace, amplitude=5pt, mirror, raise=3pt}, thick]
    (\inoxend,\sandwichbottom) -- (\backface,\sandwichbottom)
    node[midway, below=10pt, font=\small] {7 mm};

% Positions z
\draw[dashed, gray] (\frontface,0) -- (\frontface,\sandwichbottom);
\node[below, font=\scriptsize, gray] at (\frontface,-0.3) {3.75};

\draw[dashed, gray] (\backface,0) -- (\backface,\sandwichbottom);
\node[below, font=\scriptsize, gray] at (\backface,-0.3) {5.55};

% ============================================================================
% DÉTECTEUR (Sphère d'eau à z = 20 cm)
% ============================================================================
\def\detectorz{12}  % Position réduite pour le dessin (représente z=20cm)
\def\detectorradius{1.0}

\filldraw[fill=water!50, draw=blue!70!black, thick] 
    (\detectorz,3.5) circle (\detectorradius);
\node[font=\bfseries, blue!70!black] at (\detectorz,3.5) {H$_2$O};
\node[below=1.2cm, font=\small] at (\detectorz,3.5) {$z = 20$ cm};
\node[below=1.7cm, font=\small] at (\detectorz,3.5) {$r = 2$ cm};

% Flèche indiquant la distance
\draw[<->, thick, gray] (5.55,7) -- (12,7) 
    node[midway, above, font=\small] {14.45 cm};

% ============================================================================
% LÉGENDE
% ============================================================================
\node[anchor=north west] at (0,-1.5) {
    \begin{tikzpicture}[scale=0.8]
        % Légende W/PETG
        \fill[wpetg!70] (0,0) rectangle (0.5,0.4);
        \draw[black] (0,0) rectangle (0.5,0.4);
        \node[right, font=\small] at (0.6,0.2) {W/PETG 75\%/25\% ($\rho = 4.24$ g/cm$^3$, Z$_{\text{eff}} \approx 65$)};
        
        % Légende Inox
        \fill[inox] (0,-0.6) rectangle (0.5,-0.2);
        \draw[black] (0,-0.6) rectangle (0.5,-0.2);
        \node[right, font=\small] at (0.6,-0.4) {Inox 304 ($\rho = 8.0$ g/cm$^3$, Z$_{\text{eff}} \approx 26$)};
        
        % Légende Eau
        \fill[water!50] (0,-1.2) rectangle (0.5,-0.8);
        \draw[blue!70!black] (0,-1.2) rectangle (0.5,-0.8);
        \node[right, font=\small] at (0.6,-1.0) {Détecteur eau ($\rho = 1.0$ g/cm$^3$)};
        
        % Légende Source
        \filldraw[fill=source, draw=black] (0.25,-1.6) circle (0.15);
        \node[right, font=\small] at (0.6,-1.6) {Source Eu-152 (cône 60°)};
    \end{tikzpicture}
};

\end{tikzpicture}
\caption{Coupe longitudinale du système de blindage sandwich W/PETG + Inox + W/PETG. La source Eu-152 émet des gammas dans un cône de 60° vers le détecteur sphérique d'eau situé à $z = 20$ cm.}
\label{fig:coupe}
\end{figure}

%==============================================================================
\section{Caractéristiques du Sandwich}
%==============================================================================

\begin{table}[h!]
\centering
\caption{Propriétés des couches du sandwich}
\label{tab:couches}
\begin{tabular}{@{}lcccc@{}}
\toprule
\textbf{Couche} & \textbf{Matériau} & \textbf{Épaisseur} & \textbf{Densité} & \textbf{Masse surf.} \\
 & & (mm) & (g/cm$^3$) & (g/cm$^2$) \\
\midrule
1 (avant) & W/PETG 75/25 & 7.0 & 4.24 & 2.97 \\
2 (centre) & Inox 304 & 4.0 & 8.00 & 3.20 \\
3 (arrière) & W/PETG 75/25 & 7.0 & 4.24 & 2.97 \\
\midrule
\textbf{Total} & -- & \textbf{18.0} & -- & \textbf{9.14} \\
\bottomrule
\end{tabular}
\end{table}

%==============================================================================
\section{Résultats de Simulation}
%==============================================================================

\subsection{Paramètres de la simulation}

\begin{table}[h!]
\centering
\caption{Paramètres de la simulation Geant4}
\begin{tabular}{@{}ll@{}}
\toprule
\textbf{Paramètre} & \textbf{Valeur} \\
\midrule
Code & Geant4 11.03-patch-01 \\
Physique EM & Livermore (basse énergie) \\
Nombre d'événements & $25 \times 10^6$ \\
Source & Eu-152 (spectre complet 12 raies) \\
Activité & 44 kBq \\
Mode d'émission & Cône de 60° \\
Temps simulé & 568.18 s (0.158 h) \\
Facteur de correction & $f_{\text{corr}} = 0.25$ \\
\bottomrule
\end{tabular}
\end{table}

\subsection{Statistiques de génération}

\begin{table}[h!]
\centering
\caption{Validation de la génération des gammas primaires}
\begin{tabular}{@{}lcc@{}}
\toprule
\textbf{Paramètre} & \textbf{Simulé} & \textbf{Théorie} \\
\midrule
Gammas générés & 48\,102\,176 & -- \\
Moyenne $\gamma$/événement & 1.9241 & 1.924 \\
Événements avec $N_\gamma = 0$ & 11.07\% & $\sim$11.7\% \\
Gammas atteignant détecteur & 198\,786 & 588\,411 (géom.) \\
Transmission & 33.8\% & -- \\
\bottomrule
\end{tabular}
\end{table}

\subsection{Débits de dose simulés}

\begin{table}[h!]
\centering
\caption{Débits de dose dans le détecteur eau (3 méthodes)}
\label{tab:dose}
\begin{tabular}{@{}lccc@{}}
\toprule
\textbf{Méthode} & \textbf{Débit brut} & \textbf{Débit corrigé} & \textbf{Incertitude} \\
 & (nGy/h) & (nGy/h) & (nGy/h) \\
\midrule
\textbf{1 - MC direct} & 412.46 & \textbf{103.12} & $\pm$ 0.69 \\
\textbf{1bis - Forçage} & 430.01 & \textbf{107.50} & $\pm$ 0.24 \\
\textbf{2 - Fluence} & 430.01 & \textbf{107.50} & $\pm$ 0.24 \\
\midrule
\textit{Théorique (sans écran)} & -- & \textit{174.8} & -- \\
\bottomrule
\end{tabular}
\end{table}

\vspace{0.5cm}

\begin{center}
\fbox{\parbox{0.9\textwidth}{
\centering
\textbf{Résultat principal}\\[0.3cm]
\large
$\dot{D}_{\text{sandwich}} = 107.50 \pm 0.24$ nGy/h\\[0.2cm]
\normalsize
\textbf{Facteur d'atténuation :} $\displaystyle \frac{\dot{D}_{\text{sandwich}}}{\dot{D}_{\text{théo}}} = \frac{107.50}{174.8} = 0.615$\\[0.2cm]
\textbf{Atténuation :} 38.5\%
}}
\end{center}

%==============================================================================
\section{Description des Méthodes de Calcul}
%==============================================================================

\subsection{Méthode 1 : Dépôt d'énergie Monte-Carlo direct}

Somme de l'énergie déposée par toutes les particules (gammas, électrons) dans le volume d'eau :
\begin{equation}
\dot{D}_1 = \frac{E_{\text{déposée}}}{m_{\text{eau}} \times t_{\text{simulé}}} \times f_{\text{corr}}
\end{equation}

\subsection{Méthode 1bis : Forçage d'interaction}

Pour chaque gamma traversant le détecteur, calcul de l'énergie déposée théorique :
\begin{equation}
\dot{D}_{1\text{bis}} = \frac{1}{m_{\text{eau}} \times t} \sum_i E_i \times L_i \times \left(\frac{\mu_{\text{en}}}{\rho}\right)_{E_i} \times \rho_{\text{eau}} \times f_{\text{corr}}
\end{equation}
où $L_i$ est la longueur de corde du gamma $i$ dans la sphère.

\subsection{Méthode 2 : Fluence spectrale}

Équivalente à la méthode 1bis, utilisant les coefficients de conversion fluence-dose :
\begin{equation}
\dot{D}_2 = \sum_i \Phi_i \times h_K(E_i) \times f_{\text{corr}}
\end{equation}

\vspace{0.5cm}
\noindent\textit{Note : Les méthodes 1bis et 2 donnent des résultats identiques car elles utilisent la même formulation analytique. La méthode 1 (MC direct) présente plus de fluctuations statistiques.}

\end{document}
